\documentclass[11pt]{beamer}




\usepackage{booktabs}
\usetheme{CambridgeUS}
\usefonttheme{serif}
\usepackage{palatino}
\usepackage[default]{opensans}
\useinnertheme{circles}
\setbeamertemplate{caption}[numbered]
%----------------------------------------------------------------------------------------
%	PRESENTATION INFORMATION
%----------------------------------------------------------------------------------------

\title{The Discovery of an  Algebraic structure}

\subtitle{CSMI 2024}
\author[ASSIGBE Komi . RAHOUTI Chahid .]{ASSIGBE Béni \and  RAHOUTI Chahid}
\institute[]{University of Strasbourg \\ \smallskip} % Your institution, the optional parameter can be used for the institution shorthand and will appear on the bottom of every slide after author names, while the required parameter is used on the title slide and can include your email address or additional information on separate lines

\date[\today]{Mathematics and applications \\ \today} % Presentation date or conference/meeting name, the optional parameter can contain a shortened version to appear on the bottom of every slide, while the required parameter value is output to the title slide

%----------------------------------------------------------------------------------------

\begin{document}

%----------------------------------------------------------------------------------------
%	TITLE SLIDE
%----------------------------------------------------------------------------------------

\begin{frame}
	\titlepage % Output the title slide, automatically created using the text entered in the PRESENTATION INFORMATION block above
\end{frame}

%----------------------------------------------------------------------------------------
%	TABLE OF CONTENTS SLIDE
%----------------------------------------------------------------------------------------

\begin{frame}
	\frametitle{Presentation overview } % Slide title, remove this command for no title
	\tableofcontents % Display a table of contents
\end{frame}

%----------------------------------------------------------------------------------------
%	PRESENTATION BODY SLIDES
%----------------------------------------------------------------------------------------

\section{Introduction}

%------------------------------------------------

\subsection{Definition}
\begin{frame}
    \frametitle{Definition}
    What is an Algebraic structure? An algebraic structure consists of a nonempty set A (called the underlying set, carrier set or domain), a collection of operations on A (typically binary operations such as addition and multiplication), and a finite set of identities, known as axioms, that these operations must satisfy.

    Among the multiples algebraic structures, we can name:
    \begin{itemize}
        \item Group
        \item Ring
        \item Field
        \item Vector space
        \item .....
    \end{itemize}
\end{frame}

%------------------------------------------------


%---------------------------------------------------------------------------------
\section{The Problems}
\begin{frame}
    \frametitle{The Problems}
    The problems consist if we are giving a set of data in V, and those data is defined on a regular surface or variety, and we want to know if it is possible to detect a certain algebraic structure on it.
\end{frame}
\begin{frame}
		\frametitle{The First Approch}
	Let's take an example of a group structure.
	let consider a surface $M \in \mathbb{R}^{n \times n}$
	Like, 
	M = 
	$\begin{bmatrix}
		\vdots & \vdots & \vdots &\vdots \\
		x_{1} & x_{2} & \cdots & x_{n} \\
		\vdots & \vdots & \vdots &\vdots \\
	\end{bmatrix}$
	with x_{i} $\in \mathbb{R}^{d}$
	The question is if we consider M as a group structure, it is possible to find a binary operation ($\circ$) on M such that M is a group.
	\begin{enumerate}
		\item $\exists e \in M, \forall x \in M, e \circ x = x \circ e = x$
		\item $\forall{} x,y \in{} M, x \circ y = y \circ x$
		\item $\forall x,y,z \in M, (x \circ y) \circ z = x \circ (y \circ z)$
		\item $\forall x \in M, \exists -x \in M, x \circ -x = -x \circ x = e$
	\end{enumerate}
\end{frame}







%---------------------------------------------------------------------------------
\begin{frame}
    \frametitle{The Second Approch}
	Let's take an example of a group structure.
	let consider a set of points  $ V $ and let  $ f: R \rightarrow V $ a one-to-one
	function from $R$ into a codomain $V$. We define the vector addition by
	
     $$ x \oplus y = f(f^{-1}(x) + f^{-1}(y)) $$
		


	The question is if we consider $V$ as a group structure, it is possible to find a binary operation ($\oplus$) on $V$ such that this relation is satisfying.
	
\end{frame}


%---------------------------------------------------------------------------------
\section{Application}
\begin{frame}
	\frametitle{Application}
	Many differential equations encountered 
	in solving have parametric solutions.
	 Thus, to find the solution for each 
	 parameter, this often requires 
	 numerous calculations. To optimize 
	 computation and storage times, we 
	 define an algebraic structure 
	 whereby, if we compute the solution 
	 for parameters $\lambda_1$ and $ \lambda_2 $,
	  we can deduce $\lambda_3$ ...

\end{frame}






%-----------------------------------------------------------------------------------



\end{document} 