\documentclass{article}

% Packages
\usepackage[utf8]{inputenc}
\usepackage[T1]{fontenc}
\usepackage[french]{babel}
\usepackage{amssymb}
\usepackage{amsmath}
\usepackage{amsthm}
\usepackage{pgfplots}
\newtheorem{Example}{Example}

% Titre du rapport
\title{The Discovery of an  Algebraic structure}
\author{ASSIGBE Komi . RAHOUTI Chahid .}
\date{\today}

\begin{document}

\maketitle

\section{Introduction}
\subsection{Definition}
\begin{frame}
    What is an Algebraic structure? An algebraic structure consists of a nonempty set A (called the underlying set, carrier set or domain), a collection of operations on A (typically binary operations such as addition and multiplication), and a finite set of identities, known as axioms, that these operations must satisfy.

    Among the multiples algebraic structures, we can name:
    \begin{itemize}
        \item Group
        \item Ring
        \item Field
        \item Vector space
        \item .....
    \end{itemize}
\end{frame}
\begin{Example}
    A simple example of a group for 
    addition is the additive group 
    of integers $ (\mathbb{Z}, +) $ 
    satisfies the following properties 
    for any $ a, b, c \in \mathbb{Z} $:
    \begin{itemize}
        \item Closure under the addition operation: $ a + b $ is an integer.
        \item Associativity: $ (a + b) + c = a + (b + c) $.
        \item Existence of the identity element: There exists an element $ 0 \in \mathbb{Z} $ such that $ a + 0 = a $ for every $ a \in \mathbb{Z} $.
        \item Existence of inverses: For each element $ a \in \mathbb{Z} $, there exists an element $ -a \in \mathbb{Z} $ such that $ a + (-a) = 0 $.
        \item Commutativity: $ a + b = b + a $ for every $ a, b \in \mathbb{Z} $.
    \end{itemize}
These properties make $ (\mathbb{Z}, +) $ a fundamental example of an additive group.
    
\end{Example}   

\section{Objective}
\begin{frame}
Our problem is a mathematical problem known 
as data detection on structured surfaces or 
varieties. When presented with a dataset $V$ 
defined on such structured surfaces, the challenge 
arises in determining whether there exists a 
discernible algebraic pattern within the data. 
Essentially, it's about investigating whether 
there are underlying mathematical relationships 
or structures governing the given dataset. 
This problem is crucial in various fields 
such as algebraic geometry and data analysis, 
where understanding these structures aids in 
making predictions or drawing meaningful 
conclusions from the data. Thus, the goal 
is to identify and characterize the algebraic
 properties inherent in the dataset for further
  analysis and interpretation.\\

\begin{tikzpicture}
\begin{axis}[
    axis lines = middle,
    xlabel = $x$,
    ylabel = {$f(x)$},
    domain=-3:3,
    samples=100,
]

\addplot[blue, thick]{x^3 + x^2};
\addplot[red, mark = *, only marks] coordinates {(-1,0) (1,2)};
\addplot[green, mark = *, only marks] coordinates {(2,12)};
\addplot[dashed, gray] coordinates {(-1,0) (1,2) (2,12)};

\end{axis}
\end{tikzpicture}


We have three main sub-objectives to address:

\begin{enumerate}
    \item Selecting a set of points in a space or variety and defining an algebraic structure on them.
    \item Defining a vector space over this set of points using a technique based on a one-to-one function.
    \item Implementing these two approaches in Python while minimizing the losses for each axiom.
\end{enumerate}



\end{frame}

\section{First Objective}
Let's take an example of a group structure.
    let consider a surface $M \in \mathbb{R}^{n \times n}$
    Like, 
    M = 
    $\begin{bmatrix}
        \vdots & \vdots & \vdots &\vdots \\
        x_{1} & x_{2} & \cdots & x_{n} \\
        \vdots & \vdots & \vdots &\vdots \\
    \end{bmatrix}$
    with x_{i} $\in \mathbb{R}^{d}$ \\
    The question is if we consider M as a group structure, it is possible to find a binary operation ($\circ$) on M such that M is a group.
    \begin{enumerate}
        \item $\exists e \in M, \forall x \in M, e \circ x = x \circ e = x$
        \item $\forall{} x,y \in{} M, x \circ y = y \circ x$
        \item $\forall x,y,z \in M, (x \circ y) \circ z = x \circ (y \circ z)$
        \item $\forall x \in M, \exists -x \in M, x \circ -x = -x \circ x = e$
    \end{enumerate}

% Contenu de la section 2

\section{Second Objective}
	let consider a set of points  $ V $ and let  $ f: R \rightarrow V $ a one-to-one
	function from $R$ into a codomain $V$. We define the vector addition by
	
     $$ x \oplus y = f(f^{-1}(x) + f^{-1}(y)) $$
     and the scalar multiplication by
     $$ \alpha \odot x = f(  \alpha.f^{-1}(x)) $$
		
	The question is if we consider $V$ as a group 
    structure, it is possible to find a binary 
    operation ($\oplus$) and ($\otimes $) on $V$ 
    such that this relations is satisfying.
    \begin{Example}[Trivial Example]
        Let f be a function from $R$ to $Vect{(e_1)}$ such that $f(x) = x.e_1$.
        we have $f^{-1}(x) = \lambda$.
            we take x and y in $vect{(e_1)}$, we have \\
            $x\oplus y = f(f^{-1}(x) + f^{-1}(y)) = x + y$ \\
            $\alpha \odot x = f(f^{-1}(x) * \alpha) = \alpha x$ 
    \end{Example}
    \begin{Example}
        Let $\beta$ be any positive real number and let $f: \mathbb{R} \rightarrow \mathbb{R}_{+}^{*}$be
         defined by $f(x)=(1 / \beta) e^x$. Then $f$ is a one-to-one
          function from $\mathbb{R}$ onto the set of positive real numbers, 
          and $f^{-1}(x)=\ln (\beta x)$ for $x>0$. we would define vector 
          addition and scalar multiplication by
            $$
            x \oplus y=\frac{1}{\beta} e^{\ln (\beta x)+\ln (\beta y)}=\beta x y
            $$ 
            
            $$
            \alpha \odot x=\frac{1}{\beta} e^{\alpha \ln (\beta x)}=\beta^{\alpha-1} x^\alpha.
            $$ 
    \end{Example}
	
\section{losses and Implementing}
\begin{frame}
    In implementing algebraic structures, 
    the primary objective is to minimize 
    the losses associated with each axiom 
    governing these structures. 
    Here are the four losses for the first approach

    \begin{enumerate}
        \item Existence of the identity element:  \[
            L_1(\theta) = \sum_{v \in V} (e_{\text{obs}} - e_{\text{pred}})^2
            \]
        \item Commutativity:
        \[
        L_2(\theta) = \sum_{(x, y) \in V \times V} (x \oplus y - y \oplus x)^2
        \]
        \item Associativity:
        \[
        L_3(\theta) = \sum_{(x, y, z) \in V \times V \times V} ((x \oplus y) \oplus z - x \oplus (y \oplus z))^2
        \]
        \item Existence of inverses:
        \[
        L_4(\theta) = \sum_{x \in V} (-x_{\text{obs}} + (-x_{\text{pred}}))^2
        \] 
    \end{enumerate}
    Here, $\theta$ represents the parameters of our model.
     These functions $L_i(\theta)$ measure the discrepancies
      between the observed and predicted values for each group 
      axiom, where $i = 1, 2, 3, 4$. By minimizing these functions
       $L_i(\theta)$, we aim to adjust our model to be as close as
        possible to the real data, ensuring that our algebraic 
        structure accurately satisfies the group axioms. 

     $$
     L(\theta) = \min_{\theta} L_1(\theta) + L_2(\theta) + L_3(\theta) + L_4(\theta)
     $$
 This function $L(\theta)$ represents the sum of losses 
 associated with each group axiom. By minimizing 
 this function $L(\theta)$, we aim to adjust our model so
  that it optimally satisfies the four group axioms, 
  ensuring the accuracy and consistency of our 
  algebraic structure with respect to the provided data.\\
  Here are the two losses for the second approach
  \begin{enumerate}
    \item Loss for vector addition:
    \[
    L_1(\theta) = \sum_{(x, y) \in V \times V} \left\lVert x \oplus y - \left(f(f^{-1}(x)) + f(f^{-1}(y))\right) \right\rVert^2
    \]
    \item Loss for scalar multiplication:
    \[
    L_2(\theta) = \sum_{(\alpha, x) \in \mathbb{R} \times V} \left\lVert \alpha \odot x - f(\alpha \cdot f^{-1}(x)) \right\rVert^2
    \]
\end{enumerate}

Function $L(\theta)$, the sum of these two functions:
\[
L(\theta) = L_1(\theta) + L_2(\theta)
\]
Here, $\theta$ represents the parameters of 
our model. These functions $L_1(\theta)$ and
 $L_2(\theta)$ measure the discrepancies between 
 the observed and predicted values for vector addition 
 and scalar multiplication operations, respectively. 
 By minimizing these functions, we aim to adjust our 
 model to accurately represent the algebraic structure 
 present in the dataset. \\  

 We implement all of these functions 
 in Python using the PyTorch library 
 to obtain the optimal parameters $\theta$ 
 that minimize the losses and calculate 
 the accuracy of our model. To effectively
  tackle this challenge, we leverage 
  the Python programming language to 
  develop and test these algebraic 
  structures across diverse datasets




\end{frame}
\section{Application}
Many differential equations encountered 
in solving have parametric solutions.
 Thus, to find the solution for each 
 parameter, this often requires 
 numerous calculations. To optimize 
 computation and storage times, we 
 define an algebraic structure 
 whereby, if we compute the solution 
 for parameters $\lambda_1$ and $ \lambda_2 $,
 we can deduce $\lambda_3$ ...
 \begin{Example}
    Consider the general form of a first-order linear ODE with a parameter $\lambda$:
    
    \[
    \frac{dy}{dx} + \lambda y = 0
    \]
    
    The solution to this differential 
    equation depends on the parameter
     $\lambda$, the solution to this ODE is given by:
    
    \[
    y(x) = C e^{-\lambda x}
    \]
    where $C$ is the constant of integration.
    The solution on $y(0)=1$ is given by:
    $y(x) = e^{-\lambda x}$
    

\begin{tikzpicture}
\begin{axis}[
    axis lines = middle,
    xlabel = $x$,
    ylabel = {$y(x)$},
    domain=0:3,
    samples=100,
    legend pos=outer north east
]

\addplot[blue, thick]{exp(-0.5*x)}; \addlegendentry{$\lambda = 0.5$}
\addplot[red, thick]{exp(-3*x)}; \addlegendentry{$\lambda = 3$}
\addplot[green, thick]{exp(-7*x)}; \addlegendentry{$\lambda = 7$}

\end{axis}
\end{tikzpicture}





    In this example, the parameter $\lambda$ affects the behavior of the solution function $y(x)$. Different values of $\lambda$ lead to different solutions, each with its own characteristic behavior. Thus, the solution is a function with a parameter $\lambda$.
    \end{Example}

\section{Conclusion}
this project provides a comprehensive 
exploration of algebraic structures and
 their relevance in data analysis, 
 offering insights into how these 
 structures can be effectively applied 
 to detect patterns and derive meaningful
  conclusions from datasets. Through Python 
  implementation and testing, the project 
  demonstrates practical approaches to 
  optimize algebraic structures for 
  improved accuracy and performance 
  in various applications
    
\end{document}